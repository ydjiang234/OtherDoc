\documentclass[12pt]{article}
\usepackage{geometry}
%\usepackage{times}
\geometry{a4paper, top=1in, left=1in, right=1in, bottom=1.5in}
\usepackage{authblk}
\usepackage{fancyhdr}
\usepackage{graphicx}
\usepackage{tabularx}
\usepackage{colortbl}
\usepackage{tabularx}
\usepackage[labelfont={it},textfont={it}]{caption}

%define vars
\newcommand{\tit}{Intall Abaqus 2017 on a Linux System}
\newcommand{\ifp}{\textit{Ins\_File\_Dir}}
\newcommand{\cmd}[1]{
    \leavevmode{\parindent=0.05\textwidth\indent
    \begin{tabular}{>{\columncolor[gray]{0.8}}p{0.9\textwidth}}
        #1
    \end{tabular}
    }
}

%Paragraph
\setlength{\parindent}{0pt}
\renewcommand{\baselinestretch}{1.3}
\setlength{\parskip}{12pt}

%Title page
\title{\includegraphics[height=1in]{Figures/NUIG_Logo.jpg} \vspace{100pt} \\ \tit}
\author[1]{Yadong Jiang}
\affil[1]{College of Engineering and Informatics, National University of Ireland Galway}
\date{}

%Header and footer
\pagestyle{fancy}
\fancyhf{}
\lhead{\includegraphics[height=0.5in]{Figures/NUIG_Logo.jpg}}
%\rhead{Yadong Jiang}
\rhead{\tit}
\setlength{\headsep}{0.2in}
\rfoot{Page \thepage}

%Tabular
\def\arraystretch{1.1}
\begin{document}
\maketitle

\newpage

\section*{Description}
This document aims to guide users to install and run Abaqus 2017 on a Linux machine. Table \ref{tb-1} summarizes the environments of Linux machine used in this guidance. Despite Ubuntu, this instruction also applies for installation of Abaqus on other Linux distributions (e.g. Archlinux, Fedora, etc.). The Linux desktop environment is not required. All installation steps are completed in terminal. Thus, this document is also suitable for users who want to utilize Abaqus on a Linux server remotely via SSH connection. Furthermore, if you want to install other version of Abaqus, the methods given by this guide can be indicative. But it should be highlighted that the administration right is required to complete the installation.

\begin{table}[h!]
\caption{Environments of Installation}
\begin{center}
\begin{tabular}{l l}
    \hline
    Linux Distribution: & Ubuntu 14.04.4 LTS (64 bits)\\
    \hline
    Linux desktop environment: & N/A \\
    \hline
    Administration Right: & Yes \\
    \hline
    Abaqus version: & Abaqus 2017 (64 bits) \\
    \hline
\end{tabular}
\end{center}
\label{tb-1}
\end{table}

\section*{Prerequirments}
\subsection*{Installation Files}
The Abaqus installation files should be transferred to the Linux machine. In this document, the root directory of the installation files is marked \ifp. It should be noted that the \ifp\ should not contain any ``space''.

To avoid any permission issues, the permission of \ifp\ should be modified so that the root user can read/write/execute the installation files. To achieve this simply, run the following command in terminal.

\cmd{sudo chmod -R 777 \ifp}

\subsection*{Install Required Packages}
The Linux Abaqus installation program is based on ``ksh'' while the FLEXNET requires the ``lsb-core''. The two packages should be installed before starting the installation of Abaqus.

\cmd{sudo apt-get install ksh lsb-core}

It should be noted that ``apt-get'' is the package handling utility of Ubuntu. If you are working with other Linux distribution, the command of installing packages may vary.

\subsection*{Disguise Your Linux Distribution}
The Abaqus installation program will check your Linux release name. If your Linux distribution is not ``Red Hat Enterprise Linux Server'', ``Red Hat Enterprise Linux Client'', ``Red Hat Enterprise Linux Workstation'', ``Cent OS''or ``Suse Linux'', the program will pop out error and exit, as shown in Figure \ref{fig-1}.

\begin{figure}[h!]
\begin{center}
    \includegraphics[width=0.5\textwidth]{Figures/UnknowLinux.png}
\end{center}
\caption{Unknown linux release}
\label{fig-1}
\end{figure}

To avoid this error, you have to disguise your Linux distribution as one of the aforementioned Linux release names. To achieve this, you have to modified the ``Linux.sh'' file, a script which is called by the Abaqus installation program to detect your Linux release name. In this script, you have to change the line:

\cmd{``DSY\_OS\_Release =`lsb\_release --short --id |sed 's/ //g'`''}

to

\cmd{``DSY\_OS\_Release=``CentOS''''}

In terminal, you can edit text script file with software ``nano'' or ``vim''. The modification results are highlighted in Figure \ref{fig-2}. By modifying the ``Linux.sh'' script, the installation program will detect your Linux release name as ``Cent OS''.

\begin{figure}[h!]
\begin{center}
    \includegraphics[width=\textwidth]{Figures/FakeLinux.png}
\end{center}
\caption{Change the ``Linux.sh'' file}
\label{fig-2}
\end{figure}

There are eight ``Linux.sh'' scripts in the \ifp, you need to modify all the eight files. The directories which contain the scripts are listed in Table \ref{tb-2}.

\begin{table}[h!]
\caption{Directories which contain the ``Linux.sh'' files}
\begin{center}
\begin{tabular}{l}
    \hline
    \ifp/1/inst/common/init \\
    \hline
    \ifp/1/SIMULIA\_Documentation/AllOS/1/inst/common/init \\
    \hline
    \ifp/2/SIMULIA\_FLEXnet\_LicenseServer/Linux64/1/inst/common/init \\
    \hline
    \ifp/2/SIMULIA\_AbaqusServices/Linux64/1/inst/common/init \\
    \hline
    \ifp/2/SIMULIA\_AbaqusServices\_CAA\_API/Linux64/1/inst/common/init \\
    \hline
    \ifp/2/SIMULIA\_Abaqus\_CAE/Linux64/1/inst/common/init \\
    \hline
    \ifp/2/SIMULIA\_Tosca/Linux64/1/inst/common/init \\
    \hline
    \ifp/3/SIMULIA\_Isight/Linux64/1/inst/common/init \\
    \hline
\end{tabular}
\end{center}
\label{tb-2}
\end{table}

\section*{Installation}
First of all, login as root user before the installation. In Ubuntu, you can do this by:

\cmd{sudo -s}

To start the installation program, the following command should be executed:

\cmd{\ifp/1/StartTUI.sh}

This command will start a installation program in terminal. Alternatively, if you have Linux with Desktop Environment installed, you can call the following command to start the installation program with a GUI.

\cmd{\ifp/1/StartGUI.sh}

It should be highlighted that you cannot start the installation program from the \ifp\ or its subdirectories. You can switch your current directory with ``cd'' command.

When the installation program starts, you can select and install the SIMULIA products. To install Abaqus, the products ``FLEXnet License Server'', ``Abaqus Simulation Services'', ``Abaqus Simulation Services CAA API'' and ``Abaqus/CAE'' are necessary.

It should be noted that the ``FLEXnet License Server'' should be installed before installing the ``Abaqus/CAE''. When install the ``FLEXnet License Server'', you need to tick the option ``Files only :  do not start the license server program. 
''. When ``FLEXnet License Server'' is installed, you need to start the FLEXnet Server before starting installing ``Abaqus/CAE''. To achieve this, you need to open a new terminal window (or a new SSH connection) and start server as root user with the following command (if you install the ``FLEXnet License Server'' into the default directory).

\cmd{/usr/SIMULIA/License/2017/linux\_a64/code/bin/lmgrd}

During the installation of ``Abaqus/CAE'', the ``SIMULIA FLEXnet'' should be ticked as the license server. The license server of NUI Galway is ``27004@cert1.srv.nuigalway.ie''. 

After the installation, the Abaqus licence type should be changed from academic to research. To do this, one of the Abaqus environment files, ``custom\_v6.env'', should be modified. If the Abaqus is installed into its default directory, the file can be edit with the command:

\cmd{nano /usr/DassaultSystemes/SimulationServices/V6R2017x/linux\_a64/SMA/\\site/custom\_v6.env}

The line ``academic=TEACHING'' in the ``custom\_v6.env'' file should be commented or deleted, as shown in Figure \ref{fig-3}.

\begin{figure}[h!]
    \begin{center}
        \includegraphics[width=\textwidth]{Figures/Custom_v6.png}
    \end{center}
    \caption{Change the license setting in the ``custom\_v6.env'' file}
    \label{fig-3}
\end{figure}

After finishing the installation, exit the root mode with command ``exit''.

\section*{Start Abaqus}
If Abaqus is installed into its default directory, it can be called with command:

\cmd{/var/DassaultSystemes/SIMULIA/Commands/abaqus}

To avoid typing the full path to start Abaqus, the path ``/var/DassaultSystemes/SIMULIA\\/Commands'' needs to be appended to the ``PATH'' variable of Linux. The system ``PATH'' variable is defined in the file ``/etc/environment''. Open the file with ``nano'' or ``vim'' and append the Abaqus command path into the ``PATH'' variable, as shown in Figure \ref{fig-4}.

\begin{figure}[h!]
    \begin{center}
        \includegraphics[width=\textwidth]{Figures/AppendPath.png}
    \end{center}
    \caption{Append path to ``PATH'' variable}
    \label{fig-4}
\end{figure}

After modifying the ``/etc/environment'' file, a reboot or re-login is required to take the modification into effects. Now the Abaqus can be simply called by:

\cmd{abaqus}

The Abaqus CAE can be started by:

\cmd{abaqus cae}

It should be noted that other libraries (e.g. libstdc++.so.6, libjpeg62, etc.) may be required by Abaqus CAE. Just follow the error message given by Abaqus CAE and install the missing libraries with ``apt-get''.

\section*{Link C/C++ and Fortran compiler}
\subsection*{Change the default Fortran compiler}

On Linux, the default C/C++ compiler and Fortran compiler which Abaqus will search for are ``gcc'' and ``ifort'', respectively. As the usage of free version ``ifort'' is very limited, the ``gfortran'' compiler is considered as an alternative during linking. To do this, these two compilers should be installed:

\cmd{sudo apt-get install gcc gfortran}

Then, one of the Abaqus environment files, ``lnx86\_64.env'', should be modified. If you installed Abaqus into the default directory, the file is located in ``/usr/DassaultSystemes/ SimulationServices/V6R2017x/linux\_a64/SMA/site''. It is always recommended to create a backup file before editing the environment file.

To switch the default Fortran compiler, change the line:

\cmd{fortCmd = ``ifort''}

to:

\cmd{fortCmd =``gfortran''}

Correspondingly, the compiling arguments should be modified from:

\cmd{compile\_fortran = [fortCmd,\\
\hspace{95pt} '-V',\\
\hspace{95pt} '-c', '-fPIC', '-auto', '-mP2OPT\_hpo\_vec\_divbyzero=F',\\
\hspace{95pt} '-extend\_source',\\
\hspace{95pt}'-fpp', '-WB', '-I\%I']
}

to:

\cmd{compile\_fortran = [fortCmd, \\
\hspace{95pt} '-c', '-fPIC',\\
\hspace{95pt} '-I\%I']
}

Then, the linking arguments should also be changed from :

\cmd{link\_sl = [fortCmd,\\
\hspace{50pt} '-V',\\
\hspace{50pt} '-cxxlib', '-fPIC', '-threads', '-shared',\\
\hspace{50pt} '-Wl,--add-needed',\\
\hspace{50pt} '\%E', '-Wl,-soname,\%U', '-o', '\%U', '\%F', '\%A', '\%L', '\%B',\\
\hspace{50pt} '-parallel',
'-Wl,-Bdynamic', '-shared-intel']
}

to:

\cmd{link\_sl = [fortCmd,\\
\hspace{50pt} '-fPIC', '-shared',\\
\hspace{50pt} '\%E', '-Wl,-soname,\%U', '-o',\\
\hspace{50pt} '\%U', '\%F', '\%A', '\%L', '\%B',\\
\hspace{50pt} '-Wl,-Bdynamic', '-lifport', '-lifcoremt']
}

\subsection*{Verify the linking}
To verify if the linking is successful, you can try to compile and link a sample Fortran file.

\cmd{abaqus make library=testUMAT.f }

If the compilation and linking are successful, you will find two output files created, one ``*.o'' file and one ``*.so'' file (Figure \ref{fig-5}).

\begin{figure}[h!]
    \begin{center}
        \includegraphics[width=0.7\textwidth]{Figures/CompileResults.PNG}
    \end{center}
    \caption{Output files of the compilation and linking}
    \label{fig-5}
\end{figure}


\end{document}